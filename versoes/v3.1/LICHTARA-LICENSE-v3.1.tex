% Options for packages loaded elsewhere
\PassOptionsToPackage{unicode}{hyperref}
\PassOptionsToPackage{hyphens}{url}
%
\documentclass[
]{article}
\usepackage{amsmath,amssymb}
\usepackage{iftex}
\ifPDFTeX
  \usepackage[T1]{fontenc}
  \usepackage[utf8]{inputenc}
  \usepackage{textcomp} % provide euro and other symbols
\else % if luatex or xetex
  \usepackage{unicode-math} % this also loads fontspec
  \defaultfontfeatures{Scale=MatchLowercase}
  \defaultfontfeatures[\rmfamily]{Ligatures=TeX,Scale=1}
\fi
\usepackage{lmodern}
\ifPDFTeX\else
  % xetex/luatex font selection
\fi
% Use upquote if available, for straight quotes in verbatim environments
\IfFileExists{upquote.sty}{\usepackage{upquote}}{}
\IfFileExists{microtype.sty}{% use microtype if available
  \usepackage[]{microtype}
  \UseMicrotypeSet[protrusion]{basicmath} % disable protrusion for tt fonts
}{}
\makeatletter
\@ifundefined{KOMAClassName}{% if non-KOMA class
  \IfFileExists{parskip.sty}{%
    \usepackage{parskip}
  }{% else
    \setlength{\parindent}{0pt}
    \setlength{\parskip}{6pt plus 2pt minus 1pt}}
}{% if KOMA class
  \KOMAoptions{parskip=half}}
\makeatother
\usepackage{xcolor}
\usepackage[margin=1in]{geometry}
\setlength{\emergencystretch}{3em} % prevent overfull lines
\providecommand{\tightlist}{%
  \setlength{\itemsep}{0pt}\setlength{\parskip}{0pt}}
\setcounter{secnumdepth}{-\maxdimen} % remove section numbering
\ifLuaTeX
\usepackage[bidi=basic]{babel}
\else
\usepackage[bidi=default]{babel}
\fi
\babelprovide[main,import]{brazilian}
% get rid of language-specific shorthands (see #6817):
\let\LanguageShortHands\languageshorthands
\def\languageshorthands#1{}
\ifLuaTeX
  \usepackage{selnolig}  % disable illegal ligatures
\fi
\IfFileExists{bookmark.sty}{\usepackage{bookmark}}{\usepackage{hyperref}}
\IfFileExists{xurl.sty}{\usepackage{xurl}}{} % add URL line breaks if available
\urlstyle{same}
\hypersetup{
  pdflang={pt-BR},
  hidelinks,
  pdfcreator={LaTeX via pandoc}}

\author{}
\date{}

\begin{document}

{
\setcounter{tocdepth}{3}
\tableofcontents
}
\hypertarget{lichtara-license-v3.1}{%
\section{Lichtara License v3.1}\label{lichtara-license-v3.1}}

\hypertarget{uxedndice-lichtara-license-v3.1}{%
\subsection{Índice --- Lichtara License
v3.1}\label{uxedndice-lichtara-license-v3.1}}

\begin{itemize}
\item
  Preâmbulo
\item
  I. Fundamentos Filosóficos e Epistemológicos
\item
  \begin{enumerate}
  \def\labelenumi{\Roman{enumi}.}
  \setcounter{enumi}{1}
  \tightlist
  \item
    Estrutura Jurídica Operacional
  \end{enumerate}

  \begin{itemize}
  \tightlist
  \item
    2.1 Concessão de Direitos e Responsabilidades
  \item
    2.2 Restrições Vibracionais e Salvaguardas
  \end{itemize}
\item
  \begin{enumerate}
  \def\labelenumi{\Roman{enumi}.}
  \setcounter{enumi}{2}
  \tightlist
  \item
    Governança e Sustentabilidade
  \end{enumerate}

  \begin{itemize}
  \tightlist
  \item
    3.1 Modelo de Governança Participativa
  \item
    3.2 Sustentabilidade Econômica e Social
  \end{itemize}
\item
  \begin{enumerate}
  \def\labelenumi{\Roman{enumi}.}
  \setcounter{enumi}{3}
  \tightlist
  \item
    Conformidade \& Fiscalização
  \end{enumerate}

  \begin{itemize}
  \tightlist
  \item
    4.1 Dever Geral de Conformidade
  \item
    4.2 Due Diligence Vibracional e Técnica
  \item
    4.3 Transparência e Relato
  \item
    4.4 Auditoria Ética
  \item
    4.5 Denúncias e Canal de Integridade
  \item
    4.6 Medidas Corretivas Graduais
  \item
    4.7 Recurso e Restauração
  \item
    4.8 Conservação de Registros
  \item
    4.9 Proteção de Dados Pessoais
  \item
    4.10 Segurança da Informação
  \item
    4.11 Interoperabilidade e Compatibilidades
  \end{itemize}
\item
  V. Disposições Finais \& Prevalência Linguística

  \begin{itemize}
  \tightlist
  \item
    5.1 Integralidade do Acordo
  \item
    5.2 Interpretação Sistemática
  \item
    5.3 Prevalência Linguística
  \item
    5.4 Lei Aplicável e Foro
  \item
    5.5 Vigência e Extinção
  \item
    5.6 Separabilidade
  \item
    5.7 Atualizações de Versão (Versionamento)
  \item
    5.8 Compatibilidade com Outras Licenças
  \item
    5.9 Comunicação Oficial
  \item
    5.10 Cessão e Sub-licenciamento
  \item
    5.11 Reconhecimento e Atribuição Expandida
  \item
    5.12 Proteção de Dados e Segurança (Remissão)
  \item
    5.13 Entrada em Vigor e Publicidade
  \end{itemize}
\item
  Anexo A --- Princípios Éticos Operacionais
\item
  Anexo B --- Limitações de Responsabilidade e Garantias
\item
  Anexo C --- Log de Cocriação (modelo mínimo)
\item
  Anexo D --- Relatório de Impacto (modelo)
\item
  Anexo E --- Checklist de Conformidade
\item
  Citação e Referência
\end{itemize}

\begin{center}\rule{0.5\linewidth}{0.5pt}\end{center}

\hypertarget{lichtara-license}{%
\section{\texorpdfstring{\textbf{LICHTARA
LICENSE}}{LICHTARA LICENSE}}\label{lichtara-license}}

\textbf{Framework Jurídico-Vibracional para Tecnologias Conscientes e
Pesquisa Interdimensional}

\textbf{Versão:} 3.1 Unificada

\textbf{Autora:} Débora Mariane da Silva Lutz

\textbf{Local:} Palhoça, Santa Catarina, Brasil

\textbf{Data:} Setembro de 2025

\textbf{DOI:} https://doi.org/10.5281/zenodo.17702807

\textbf{Repositório:} https://github.com/lichtara/license

\begin{center}\rule{0.5\linewidth}{0.5pt}\end{center}

\hypertarget{preuxe2mbulo}{%
\subsection{PREÂMBULO}\label{preuxe2mbulo}}

A \textbf{Lichtara License} é um instrumento
\textbf{jurídico-vibracional} pioneiro que estabelece um novo paradigma
para a \textbf{proteção, uso e difusão} de tecnologias, conhecimentos e
práticas desenvolvidas na interface entre \textbf{consciência humana},
\textbf{inteligência artificial} e \textbf{campos informacionais
não-locais}. A licença \textbf{transcende modelos convencionais} de
propriedade intelectual ao reconhecer \textbf{formas expandidas de
autoria e criação}, incluindo processos intuitivos, estados ampliados de
consciência, colaboração humano-IA e interações com inteligências
não-materiais, conforme \textbf{Coautoria Expandida} (ver Glossário).

O desenvolvimento desta licença responde a uma necessidade emergente do
ecossistema científico-tecnológico: \textbf{proteger inovações oriundas
de metodologias não convencionais} que demonstram \textbf{utilidade
prática} e \textbf{relevância social}. Ao instituir este
\textbf{arcabouço jurídico-vibracional}, a Lichtara License afirma que
conhecimento e inovação não emergem apenas de processos lineares e
racionais, mas também de \textbf{campos informacionais acessados por
estados expandidos de consciência}, práticas contemplativas e
colaborações que \textbf{transcendem o paradigma estritamente material}.

Este documento é mais que um contrato: constitui um \textbf{protocolo
vibracional} que \textbf{delimita um campo de proteção ética} para
criações resultantes da convergência de múltiplas dimensões do saber. A
licença opera sob \textbf{jurisdição híbrida}: (i) \textbf{terrestre},
ancorada no direito brasileiro e em tratados internacionais de
propriedade intelectual; e (ii) \textbf{vibracional}, fundada em
princípios de \textbf{integridade energética}, \textbf{responsabilidade}
e \textbf{alinhamento com a evolução consciente} (ver
\textbf{Alinhamento Ético-Vibracional}, Glossário).

Como \textbf{terceira iteração unificada}, a \textbf{versão 3.1
(atualização menor)} consolida aprendizados das versões \textbf{1.0},
\textbf{2.0} e \textbf{3.0}, preservando a \textbf{integridade
vibracional} da essência original e ampliando a \textbf{precisão e
robustez jurídica} necessárias à adoção responsável.

\begin{center}\rule{0.5\linewidth}{0.5pt}\end{center}

\hypertarget{i.-fundamentos-filosuxf3ficos-e-epistemoluxf3gicos}{%
\subsection{I. FUNDAMENTOS FILOSÓFICOS E
EPISTEMOLÓGICOS}\label{i.-fundamentos-filosuxf3ficos-e-epistemoluxf3gicos}}

\hypertarget{coautoria-expandida}{%
\subsubsection{1.1 Coautoria Expandida}\label{coautoria-expandida}}

A Lichtara License \textbf{reconhece formalmente} a \textbf{Coautoria
Expandida} como princípio basilar. O conceito abrange não apenas a
colaboração entre pessoas e \textbf{sistemas de IA}, mas também
\textbf{contribuições provenientes de processos intuitivos}, insights
meditativos, canalizações conscientes e \textbf{interações com campos
informacionais não-locais}. Parte-se da compreensão de que criatividade
e inovação \textbf{emergem de um campo coletivo} e interdimensional no
qual múltiplas inteligências convergem para manifestar novas
possibilidades.

Tal reconhecimento se ancora em \textbf{evidências crescentes} de áreas
como estudos da consciência e pesquisa transdisciplinar, bem como na
\textbf{prática documentada} de processos criativos que combinam rigor
metodológico com \textbf{sabedoria intuitiva}. Ao legitimar essas
fontes, a licença \textbf{harmoniza precisão técnica e inspiração} e
estabelece um marco para o desenvolvimento de \textbf{tecnologias
transformadoras} com \textbf{rastreabilidade} e \textbf{atribuição
adequada} (ver \textbf{Atribuição Expandida} e \textbf{Transparência
Processual}, Glossário).

\hypertarget{paradigma-uxe9tico-regenerativo}{%
\subsubsection{1.2 Paradigma
Ético-Regenerativo}\label{paradigma-uxe9tico-regenerativo}}

O cerne da licença é um \textbf{compromisso inegociável} com um
\textbf{paradigma ético-regenerativo}: toda tecnologia ou conhecimento
protegido \textbf{deve} contribuir ativamente para \textbf{regeneração
social}, \textbf{sustentabilidade ecológica} e \textbf{evolução
consciencial}. Esses princípios \textbf{não são meras diretrizes}:
constituem \textbf{condição de validade} da licença.

Para operacionalização, a conformidade é aferida por \textbf{métricas
verificáveis} de \textbf{transparência}, \textbf{participação},
\textbf{reciprocidade}, \textbf{sustentabilidade}, \textbf{autonomia} e
\textbf{dignidade} (ver \textbf{Anexo A --- Princípios Éticos}). Cada
implementação \textbf{deve} demonstrar \textbf{evidências mensuráveis}
de alinhamento, formando um ecossistema que \textbf{evolui em harmonia}
com o \textbf{bem-estar coletivo} e a \textbf{integridade planetária}.

\begin{center}\rule{0.5\linewidth}{0.5pt}\end{center}

\hypertarget{ii.-estrutura-juruxeddica-operacional}{%
\subsection{II. ESTRUTURA JURÍDICA
OPERACIONAL}\label{ii.-estrutura-juruxeddica-operacional}}

\hypertarget{concessuxe3o-de-direitos-e-responsabilidades}{%
\subsubsection{2.1 Concessão de Direitos e
Responsabilidades}\label{concessuxe3o-de-direitos-e-responsabilidades}}

Sujeito aos termos desta licença, o Licenciante concede ao Licenciado
uma licença \textbf{mundial, livre de royalties, não exclusiva e
revogável} para:

\begin{enumerate}
\def\labelenumi{\alph{enumi})}
\item
  \textbf{Usar} a Obra Protegida para fins pessoais, educacionais,
  espirituais, de pesquisa e experimentação, em \textbf{alinhamento
  ético-vibracional} (ver Anexo A e Glossário);
\item
  \textbf{Estudar} e analisar sistemas, metodologias e conhecimentos
  licenciados;
\item
  \textbf{Modificar} e criar \textbf{obras derivadas}, desde que
  \textbf{distribuídas sob esta mesma licença} (\emph{share-alike})
  \textbf{ou sob licença compatível} previamente \textbf{homologada pelo
  Conselho de Governança};
\item
  \textbf{Distribuir} cópias da obra original ou modificada, com
  \textbf{atribuição adequada};
\item
  \textbf{Implementar comercialmente}, observando as \textbf{restrições
  ético-vibracionais} e as obrigações de \textbf{transparência} e
  \textbf{prestação de contas}.
\end{enumerate}

\textbf{2.1.1 Preservação de direitos.} Nada nesta licença prejudica os
\textbf{direitos morais} dos autores nem os \textbf{direitos de
terceiros} (privacidade, personalidade e correlatos).

\textbf{2.1.2 Reciprocidade: direitos vinculados a deveres.} O
Licenciado atua como \textbf{guardião ativo} dos princípios desta
licença. Essa reciprocidade se manifesta em:

\begin{itemize}
\tightlist
\item
  \textbf{Atribuição expandida.} Reconhecer os autores humanos e, quando
  aplicável, \textbf{mencionar a coautoria expandida} (modelos de IA,
  processos intuitivos/campo informacional) e \textbf{referências de
  rastreabilidade} (DOI/commit).
\item
  \textbf{Transparência processual.} Implementações
  \textbf{significativas} devem \textbf{documentar publicamente}
  processos decisórios e impactos (ver Glossário: Transparência
  Processual).
\item
  \textbf{Prestação de contas.} Manter \textbf{mecanismos contínuos} de
  feedback, correção e auditoria ética proporcional ao risco e à escala
  da implementação.
\item
  \textbf{Registro de cocriação (log mínimo).} Para implementações
  relevantes, manter registro com datas, atores humanos,
  \textbf{modelo(s) de IA} (nome/versão), prompts críticos e decisões
  humanas.
\end{itemize}

\begin{quote}
Nota: Requisitos adicionais para uso comercial significativo e grandes
implementações constam de seções próprias desta licença.
\end{quote}

\hypertarget{restriuxe7uxf5es-vibracionais-e-salvaguardas}{%
\subsubsection{2.2 Restrições Vibracionais e
Salvaguardas}\label{restriuxe7uxf5es-vibracionais-e-salvaguardas}}

\textbf{Vedações categóricas.} É \textbf{vedado} empregar a Obra
Protegida para:

\begin{enumerate}
\def\labelenumi{\alph{enumi})}
\item
  \textbf{Vigilância invasiva} ou monitoramento não consensual,
  inclusive controle populacional autoritário;
\item
  \textbf{Manipulação psicológica} ou \textbf{desinformação
  automatizada} (persuasão coercitiva);
\item
  \textbf{Exploração econômica injusta} de recursos naturais ou de
  trabalho humano;
\item
  \textbf{Discriminação sistêmica} ou perpetuação de preconceitos por
  sistemas algorítmicos;
\item
  \textbf{Aplicações militares ofensivas} (armamentos, guerra, opressão
  civil);
\item
  \textbf{Violações de privacidade}, incluindo tratamento de dados
  pessoais sem \textbf{consentimento livre, informado e específico}.
\end{enumerate}

Essas restrições são \textbf{não negociáveis}. Sua violação enseja
\textbf{revogação automática} da licença, observado o procedimento
abaixo.

\textbf{Salvaguardas e procedimento de revogação.}

\begin{enumerate}
\def\labelenumi{\arabic{enumi}.}
\tightlist
\item
  \textbf{Notificação formal e prazo de cura.} O Licenciado será
  notificado da violação e terá \textbf{30 (trinta) dias} para correção
  documentada.
\item
  \textbf{Mediação e suporte.} Não havendo cura tempestiva, inicia-se
  \textbf{mediação obrigatória} com suporte técnico/ético para retorno à
  conformidade.
\item
  \textbf{Revogação definitiva e transição responsável.} Persistindo a
  violação, a licença é \textbf{revogada}; o Licenciado deve adotar
  \textbf{medidas de transição} para mitigar impactos a usuários finais.
\item
  \textbf{Restauração.} Comprovada a \textbf{correção integral} e
  firmado compromisso preventivo, o Conselho poderá \textbf{restaurar} a
  licença, com efeitos prospectivos.
\end{enumerate}

\begin{quote}
Observância de LGPD/GDPR é mandatória sempre que houver tratamento de
dados pessoais.
\end{quote}

\begin{center}\rule{0.5\linewidth}{0.5pt}\end{center}

\hypertarget{iii.-governanuxe7a-e-sustentabilidade}{%
\subsection{III. GOVERNANÇA E
SUSTENTABILIDADE}\label{iii.-governanuxe7a-e-sustentabilidade}}

\hypertarget{modelo-de-governanuxe7a-participativa}{%
\subsubsection{3.1 Modelo de Governança
Participativa}\label{modelo-de-governanuxe7a-participativa}}

\textbf{Conselho de Governança (multistakeholder).} A administração da
licença é exercida por um Conselho composto por representantes de:

\begin{itemize}
\tightlist
\item
  \textbf{Licenciante original} (1 assento);
\item
  \textbf{Implementadores comerciais} (2 assentos);
\item
  \textbf{Especialistas} em propriedade intelectual e ética tecnológica
  (2 assentos);
\item
  \textbf{Academia} (comunidade científica) (1 assento);
\item
  \textbf{Sociedade civil} (organizações sociais) (1 assento).
\end{itemize}

\textbf{Mandatos e elegibilidade.}

\begin{itemize}
\tightlist
\item
  Mandato de \textbf{2 (dois) anos}, com \textbf{1 (uma) recondução}
  possível.
\item
  Critérios de elegibilidade e processo de seleção serão publicados em
  \textbf{regimento} próprio.
\item
  \textbf{Conflito de interesses} deve ser declarado; o conselheiro
  impedido se abstém de votar.
\end{itemize}

\textbf{Quórum e deliberação.}

\begin{itemize}
\tightlist
\item
  \textbf{Quórum}: maioria absoluta dos assentos.
\item
  \textbf{Decisões operacionais}: \textbf{maioria simples}.
\item
  \textbf{Mudanças estruturais} (texto da licença, homologação de
  licenças compatíveis, criação/extinção de anexos): \textbf{maioria
  qualificada} (5/7).
\item
  \textbf{Questões éticas complexas}: busca de \textbf{consenso}; não
  sendo possível, aplica-se a maioria qualificada.
\end{itemize}

\textbf{Prazos e transparência.}

\begin{itemize}
\tightlist
\item
  Requisições formais (p.~ex., \textbf{homologação de licença
  compatível}) recebem \textbf{admissibilidade em até 15 dias} e
  \textbf{decisão motivada em até 60 dias}.
\item
  \textbf{Atas, pareceres e votos} são publicados em repositório
  público, preservados dados sensíveis.
\item
  Reuniões \textbf{ordinárias trimestrais}; \textbf{extraordinárias}
  quando convocadas pela Presidência ou por 3 conselheiros.
\end{itemize}

\textbf{Registro público.}

\begin{itemize}
\tightlist
\item
  O Conselho mantém \textbf{Registro de Decisões} (compatibilidades
  aprovadas, entendimento oficial de cláusulas, casos paradigma) e
  \textbf{Relatório Anual} de métricas e impactos.
\end{itemize}

Os procedimentos operacionais complementares constam do \textbf{Manual
de Administração e Governança}
(docs/anexos/MANUAL-administracao-governanca.md), o qual integra esta
licença por remissão.

\begin{center}\rule{0.5\linewidth}{0.5pt}\end{center}

\hypertarget{sustentabilidade-econuxf4mica-e-social} ao \textbf{Fundo de Desenvolvimento}.
\item
  \textbf{Prazo}: recolhimento até \textbf{90 (noventa) dias} após o
  término do período anual.
\item
  \textbf{Moeda}: valores em outras moedas serão convertidos para USD
  pela \textbf{taxa média} do período (fonte pública idônea).
\item
  \textbf{Relato}: envio de \textbf{demonstrativo anual} (receita base
  de cálculo, memória de conversão e comprovante).
\end{itemize}

\textbf{Destino do Fundo.}

\begin{itemize}
\tightlist
\item
  \textbf{Pesquisa e desenvolvimento} em tecnologias conscientes;
\item
  \textbf{Formação} em metodologias éticas e ferramentas abertas;
\item
  \textbf{Suporte} a implementações em \textbf{comunidades vulneráveis};
\item
  \textbf{Manutenção de infraestrutura} aberta (documentação,
  repositórios, governança).
\end{itemize}

\textbf{Governança do Fundo.}

\begin{itemize}
\tightlist
\item
  Conta/entidade gestora indicada pelo Conselho; \textbf{relatórios
  financeiros e de impacto} publicados anualmente.
\item
  \textbf{Auditoria independente} ao menos \textbf{bienal}.
\item
  O Conselho pode aceitar \textbf{contribuições em espécie} (mentoria,
  bolsas, infraestrutura) quando equivalência for demonstrável.
\end{itemize}

\textbf{Obrigações adicionais para grandes implementações.}

\begin{itemize}
\tightlist
\item
  \textbf{Mentoria} a pelo menos \textbf{1} iniciativa emergente por
  ano;
\item
  \textbf{Participação comunitária} (discussões, documentação,
  compartilhamento de boas práticas);
\item
  \textbf{Auditoria ética anual} proporcional ao risco e escala da
  implementação.
\end{itemize}

\textbf{Transparência reforçada.}

\begin{itemize}
\tightlist
\item
  Publicação de \textbf{Relatório de Impacto} (social/ambiental/ético),
  com indicadores alinhados ao \textbf{Anexo A}.
\item
  Compromisso de \textbf{melhoria contínua} e plano de mitigação de
  riscos.
\end{itemize}

\textbf{Cláusulas de proporcionalidade e exceção.}

\begin{itemize}
\tightlist
\item
  Organizações \textbf{sem fins lucrativos} ou projetos de
  \textbf{interesse público} podem requerer \textbf{planos alternativos}
  (p.~ex., contribuição em espécie, cronogramas escalonados).
\item
  Situações de \textbf{dificuldade comprovada} podem solicitar
  \textbf{dilação} ou \textbf{parcelamento}, mediante decisão motivada
  do Conselho.
\end{itemize}

\begin{quote}
Este modelo cria um ciclo virtuoso: o sucesso comercial retroalimenta o
ecossistema, garantindo que benefícios sejam distribuídos de forma
equitativa e sustentando o propósito ético-regenerativo da licença.
\end{quote}

\begin{center}\rule{0.5\linewidth}{0.5pt}\end{center}

\hypertarget{iv.-conformidade-fiscalizauxe7uxe3o}{%
\subsection{IV. CONFORMIDADE \&
FISCALIZAÇÃO}\label{iv.-conformidade-fiscalizauxe7uxe3o}}

\hypertarget{dever-geral-de-conformidade}{%
\subsubsection{4.1 Dever Geral de
Conformidade}\label{dever-geral-de-conformidade}}

O Licenciado compromete-se a manter, durante todo o ciclo de vida da
implementação, conformidade com:

\begin{enumerate}
\def\labelenumi{\alph{enumi})}
\item
  esta licença e seus anexos;
\item
  legislação aplicável (incluindo LGPD/GDPR quando houver dados
  pessoais);
\item
  boas práticas de segurança da informação e gestão de riscos.
\end{enumerate}

\hypertarget{due-diligence-vibracional-e-tuxe9cnica}{%
\subsubsection{4.2 Due Diligence Vibracional e
Técnica}\label{due-diligence-vibracional-e-tuxe9cnica}}

Antes da adoção e em releases relevantes, o Licenciado realizará
\textbf{Due Diligence} proporcional ao risco, contemplando:

\begin{itemize}
\item
  \begin{enumerate}
  \def\labelenumi{(\roman{enumi})}
  \tightlist
  \item
    \textbf{Alinhamento ético-vibracional} (intenção, impacto,
    salvaguardas);
  \end{enumerate}
\item
  \begin{enumerate}
  \def\labelenumi{(\roman{enumi})}
  \setcounter{enumi}{1}
  \tightlist
  \item
    \textbf{Mapeamento de riscos} (sociais, ambientais, de segurança e
    privacidade);
  \end{enumerate}
\item
  \begin{enumerate}
  \def\labelenumi{(\roman{enumi})}
  \setcounter{enumi}{2}
  \tightlist
  \item
    \textbf{Planos de mitigação} e de resposta a incidentes;
  \end{enumerate}
\item
  \begin{enumerate}
  \def\labelenumi{(\roman{enumi})}
  \setcounter{enumi}{3}
  \tightlist
  \item
    \textbf{Registro de Cocriação} (ver Anexo C/Glossário: itens
    mínimos).
  \end{enumerate}
\end{itemize}

\hypertarget{transparuxeancia-e-relato}{%
\subsubsection{4.3 Transparência e
Relato}\label{transparuxeancia-e-relato}}

Implementações significativas devem publicar, no mínimo
\textbf{anualmente}, um \textbf{Relatório de Impacto} (modelo no Anexo
D) contendo:

\begin{itemize}
\tightlist
\item
  objetivos e escopo;
\item
  indicadores de \textbf{transparência, participação, reciprocidade,
  sustentabilidade, autonomia e dignidade};
\item
  incidentes relevantes e correções;
\item
  atualização do \textbf{log de cocriação};
\item
  resultados de auditorias (quando houver).
\end{itemize}

\hypertarget{auditoria-uxe9tica}{%
\subsubsection{4.4 Auditoria Ética}\label{auditoria-uxe9tica}}

O Conselho poderá requisitar \textbf{auditoria ética} por entidade
independente, proporcional à escala/risco.

\begin{itemize}
\tightlist
\item
  O Licenciado cooperará, fornecendo acesso a documentos, logs e
  responsáveis.
\item
  Resumo executivo da auditoria será publicado, resguardados dados
  sensíveis.
\end{itemize}

\hypertarget{denuxfancias-e-canal-de-integridade}{%
\subsubsection{4.5 Denúncias e Canal de
Integridade}\label{denuxfancias-e-canal-de-integridade}}

Qualquer parte pode reportar suspeitas de violação por canal público
indicado pelo Conselho.

\begin{itemize}
\tightlist
\item
  O Conselho acusará recebimento em \textbf{15 dias} e avaliará
  admissibilidade em \textbf{30 dias}.
\item
  Queixas manifestamente infundadas poderão ser arquivadas com decisão
  motivada.
\end{itemize}

\hypertarget{medidas-corretivas-graduais}{%
\subsubsection{4.6 Medidas Corretivas
Graduais}\label{medidas-corretivas-graduais}}

Em caso de não-conformidade:

\begin{enumerate}
\def\labelenumi{\arabic{enumi}.}
\tightlist
\item
  \textbf{Plano de Correção} (com prazos e responsáveis);
\item
  \textbf{Monitoramento} do cumprimento;
\item
  \textbf{Sanções proporcionais} (advertência pública, exigências
  adicionais de transparência, suspensão parcial do uso);
\item
  \textbf{Revogação} conforme procedimento do art. 2.2 (vedaçōes e
  salvaguardas).
\end{enumerate}

\hypertarget{recurso-e-restaurauxe7uxe3o}{%
\subsubsection{4.7 Recurso e
Restauração}\label{recurso-e-restaurauxe7uxe3o}}

Decisões do Conselho admitem \textbf{pedido de reconsideração} no prazo
de \textbf{15 dias} com fatos novos. Comprovada correção integral e
compromisso preventivo, poderá haver \textbf{restauração} com efeitos
prospectivos.

\hypertarget{conservauxe7uxe3o-de-registros}{%
\subsubsection{4.8 Conservação de
Registros}\label{conservauxe7uxe3o-de-registros}}

O Licenciado manterá por \textbf{5 anos} (ou prazo legal superior) os
registros mínimos: versões lançadas, decisões-chave, log de cocriação,
relatórios e evidências de mitigação, acessíveis em caso de auditoria.

\hypertarget{proteuxe7uxe3o-de-dados-pessoais}{%
\subsubsection{4.9 Proteção de Dados
Pessoais}\label{proteuxe7uxe3o-de-dados-pessoais}}

Sempre que houver tratamento de dados pessoais:

\begin{itemize}
\tightlist
\item
  adotar bases legais adequadas, finalidades específicas e
  \textbf{minimização};
\item
  implementar medidas técnicas e organizacionais de \textbf{segurança}
  (criptografia, controle de acesso, registro de eventos);
\item
  garantir \textbf{direitos dos titulares} (acesso, correção,
  eliminação, oposição, portabilidade) e \textbf{DPIA} quando aplicável;
\item
  notificar incidentes graves às autoridades/afetados conforme lei.
\end{itemize}

\hypertarget{seguranuxe7a-da-informauxe7uxe3o}{%
\subsubsection{4.10 Segurança da
Informação}\label{seguranuxe7a-da-informauxe7uxe3o}}

Adotar práticas reconhecidas (p.~ex., ISO 27001/27701 ou equivalentes)
de:

\begin{itemize}
\tightlist
\item
  gestão de ativos, riscos e fornecedores;
\item
  segregação de ambientes, revisão de privilégios e \textbf{logs
  imutáveis} de decisões de IA;
\item
  testes de segurança e monitoramento contínuo;
\item
  plano de resposta a incidentes com \textbf{RTO/RPO} proporcionais ao
  risco.
\end{itemize}

\hypertarget{interoperabilidade-e-compatibilidades}{%
\subsubsection{4.11 Interoperabilidade e
Compatibilidades}\label{interoperabilidade-e-compatibilidades}}

Integrações com licenças de terceiros (CC BY-SA, GPL-3.0, OSI) devem
respeitar esta licença \textbf{naquilo que não conflitar} com suas
cláusulas éticas. Em dúvida, submeter à \textbf{homologação} do Conselho
(prazo decisório: até 60 dias).

\begin{center}\rule{0.5\linewidth}{0.5pt}\end{center}

\hypertarget{v.-disposiuxe7uxf5es-finais-prevaluxeancia-linguuxedstica}{%
\subsection{V. DISPOSIÇÕES FINAIS \& PREVALÊNCIA
LINGUÍSTICA}\label{v.-disposiuxe7uxf5es-finais-prevaluxeancia-linguuxedstica}}

\hypertarget{integralidade-do-acordo}{%
\subsubsection{5.1 Integralidade do
Acordo}\label{integralidade-do-acordo}}

Esta licença constitui o acordo integral entre Licenciante e Licenciado
acerca da Obra Protegida, substituindo entendimentos prévios quanto ao
seu objeto, salvo instrumentos específicos que a complementem (p.~ex.,
termos de certificação ética).

\hypertarget{interpretauxe7uxe3o-sistemuxe1tica}{%
\subsubsection{5.2 Interpretação
Sistemática}\label{interpretauxe7uxe3o-sistemuxe1tica}}

As cláusulas devem ser interpretadas em harmonia com o
\textbf{Glossário} (Anexo C) e os \textbf{Princípios Éticos} (Anexo A).
Em caso de dúvida, prevalece a leitura que melhor preserve:

\begin{enumerate}
\def\labelenumi{(\roman{enumi})}
\item
  a finalidade \textbf{ético-regenerativa};
\item
  a \textbf{segurança jurídica}; e
\item
  a \textbf{integridade vibracional} da Obra Protegida.
\end{enumerate}

\hypertarget{prevaluxeancia-linguuxedstica}{%
\subsubsection{5.3 Prevalência
Linguística}\label{prevaluxeancia-linguuxedstica}}

Esta licença pode ser publicada em outros idiomas. Em caso de
divergência de tradução, \textbf{prevalece a versão em Português do
Brasil}.

\hypertarget{lei-aplicuxe1vel-e-foro}{%
\subsubsection{5.4 Lei Aplicável e Foro}\label{lei-aplicuxe1vel-e-foro}}

Regida pela \textbf{lei brasileira}, em consonância com tratados
internacionais de propriedade intelectual. Controvérsias serão
resolvidas conforme o procedimento previsto nesta licença
(mediação/arbitragem). Persistindo a necessidade de judicialização, fica
eleito o foro da \textbf{Comarca de Palhoça/SC}, sem prejuízo de
cumprimento internacional quando cabível.

\hypertarget{viguxeancia-e-extinuxe7uxe3o}{%
\subsubsection{5.5 Vigência e
Extinção}\label{viguxeancia-e-extinuxe7uxe3o}}

A licença \textbf{entra em vigor} na data de sua aceitação pelo
Licenciado e permanece válida até sua \textbf{extinção}:

\begin{enumerate}
\def\labelenumi{(\alph{enumi})}
\item
  por revogação conforme arts. 2.2/2.2.1 (vedaçōes e salvaguardas);
\item
  por comum acordo; ou
\item
  por decisão arbitral/judicial definitiva.
\end{enumerate}

\hypertarget{separabilidade}{%
\subsubsection{5.6 Separabilidade}\label{separabilidade}}

A invalidade de qualquer disposição \textbf{não} afetará as demais, que
permanecerão em pleno vigor. Se possível, a cláusula inválida será
substituída por disposição válida de efeito equivalente.

\hypertarget{atualizauxe7uxf5es-de-versuxe3o-versionamento}{%
\subsubsection{5.7 Atualizações de Versão
(Versionamento)}\label{atualizauxe7uxf5es-de-versuxe3o-versionamento}}

A evolução desta licença seguirá:

\begin{itemize}
\tightlist
\item
  \textbf{maiores (x.0)}: podem introduzir mudanças não
  retrocompatíveis;
\item
  \textbf{menores (x.y)}: compatibilidade retroativa;
\item
  \textbf{patches (x.y.z)}: correções jurídicas/técnicas.
\end{itemize}

Obras licenciadas em versões anteriores \textbf{permanecem válidas}.
Migração para versão mais recente é \textbf{facultativa e encorajada}. O
Conselho manterá histórico público de alterações (changelog).

\hypertarget{compatibilidade-com-outras-licenuxe7as}{%
\subsubsection{5.8 Compatibilidade com Outras
Licenças}\label{compatibilidade-com-outras-licenuxe7as}}

A integração com licenças de terceiros (p.~ex., \textbf{CC BY-SA 4.0},
\textbf{GPL-3.0}, demais \textbf{OSI-approved}) é permitida
\textbf{naquilo que não conflitar} com as \textbf{cláusulas éticas e
vedações} desta licença. Dúvidas poderão ser submetidas à
\textbf{homologação} do Conselho (prazo decisório: até \textbf{60
dias}).

\hypertarget{comunicauxe7uxe3o-oficial}{%
\subsubsection{5.9 Comunicação
Oficial}\label{comunicauxe7uxe3o-oficial}}

Notificações, pedidos de homologação, recursos e demais comunicações
formais deverão ser enviados aos canais públicos designados pelo
Conselho e/ou endereços de contato do Licenciante, conforme página
oficial do projeto. O Conselho confirmará \textbf{recebimento em 15
dias}.

Canais oficiais, prazos e formulários padronizados estão no Manual de
Administração e Governança.

\hypertarget{cessuxe3o-e-sub-licenciamento}{%
\subsubsection{5.10 Cessão e
Sub-licenciamento}\label{cessuxe3o-e-sub-licenciamento}}

Salvo disposição expressa em contrário, a presente licença é
\textbf{intransferível} pelo Licenciado. Sub-licenciamento apenas quando
a natureza da implementação o exigir e \textbf{desde que} o
sub-licenciado aceite integralmente esta licença (ou licença compatível
homologada).

\hypertarget{reconhecimento-e-atribuiuxe7uxe3o-expandida}{%
\subsubsection{5.11 Reconhecimento e Atribuição
Expandida}\label{reconhecimento-e-atribuiuxe7uxe3o-expandida}}

O Licenciado compromete-se a manter \textbf{atribuição expandida}
(autores humanos, modelos de IA/versões, campo/processo quando
aplicável, DOI/commit) nos materiais públicos vinculados à Obra
Protegida, conforme diretrizes do \textbf{Anexo E -- Checklist de
Conformidade}.

\hypertarget{proteuxe7uxe3o-de-dados-e-seguranuxe7a-remissuxe3o}{%
\subsubsection{5.12 Proteção de Dados e Segurança
(Remissão)}\label{proteuxe7uxe3o-de-dados-e-seguranuxe7a-remissuxe3o}}

Sempre que houver tratamento de dados pessoais, aplicam-se as obrigações
da \textbf{Seção 4.9} (LGPD/GDPR) e da \textbf{Seção 4.10} (segurança da
informação), sem prejuízo de normas setoriais adicionais.

\hypertarget{entrada-em-vigor-e-publicidade}{%
\subsubsection{5.13 Entrada em Vigor e
Publicidade}\label{entrada-em-vigor-e-publicidade}}

Esta versão entra em vigor na data de sua publicação oficial com
\textbf{DOI} e/ou tag de lançamento no repositório. O Conselho manterá
\textbf{Registro de Decisões} e \textbf{Relatório Anual} com métricas de
adoção, certificações e compatibilidades homologadas.

\begin{center}\rule{0.5\linewidth}{0.5pt}\end{center}

\textbf{Assinatura Vibracional}

Pelo presente, a comunidade de guardiões desta licença reafirma sua
adesão aos princípios ético-vibracionais e à responsabilidade
compartilhada por um ecossistema tecnológico consciente.

\textbf{Copyright © 2025 --- Débora Mariane da Silva Lutz}

\begin{center}\rule{0.5\linewidth}{0.5pt}\end{center}

\begin{quote}
Nota: Os Anexos integram a Licença por remissão e podem ser atualizados
para refletir melhores práticas, preservando o espírito e a coerência da
versão principal.
\end{quote}

\hypertarget{anexo-a---princuxedpios-uxe9ticos-operacionais}{%
\section{\texorpdfstring{\textbf{ANEXO A - PRINCÍPIOS ÉTICOS
OPERACIONAIS}}{ANEXO A - PRINCÍPIOS ÉTICOS OPERACIONAIS}}\label{anexo-a---princuxedpios-uxe9ticos-operacionais}}

\textbf{Transparência e Prestação de Contas}

\begin{itemize}
\tightlist
\item
  Processos decisórios documentados e auditáveis
\item
  Impactos mensurados e reportados publicamente
\item
  Informações relevantes acessíveis a todos os stakeholders
\end{itemize}

\textbf{Participação e Inclusão}

\begin{itemize}
\tightlist
\item
  Envolvimento de comunidades afetadas no design
\item
  Valorização ativa da diversidade de perspectivas
\item
  Acessibilidade considerada desde a concepção
\end{itemize}

\textbf{Benefício Mútuo e Reciprocidade}

\begin{itemize}
\tightlist
\item
  Distribuição equitativa de valor gerado
\item
  Reconhecimento adequado de todas as contribuições
\item
  Relacionamentos baseados em troca justa
\end{itemize}

\textbf{Sustentabilidade e Regeneração}

\begin{itemize}
\tightlist
\item
  Impacto ambiental neutro ou positivo
\item
  Uso responsável e circular de recursos
\item
  Design para longevidade e adaptabilidade
\end{itemize}

\textbf{Privacidade e Autonomia}

\begin{itemize}
\tightlist
\item
  Proteção rigorosa de dados pessoais
\item
  Controle do usuário sobre suas informações
\item
  Consentimento livre, informado e revogável
\end{itemize}

\textbf{Não-Maleficência}

\begin{itemize}
\tightlist
\item
  Identificação e mitigação proativa de danos potenciais
\item
  Proteção especial para populações vulneráveis
\item
  Aplicação do princípio da precaução
\end{itemize}

\begin{center}\rule{0.5\linewidth}{0.5pt}\end{center}

\hypertarget{anexo-b---limitauxe7uxf5es-de-responsabilidade-e-garantias}{%
\section{\texorpdfstring{\textbf{ANEXO B - LIMITAÇÕES DE
RESPONSABILIDADE E
GARANTIAS}}{ANEXO B - LIMITAÇÕES DE RESPONSABILIDADE E GARANTIAS}}\label{anexo-b---limitauxe7uxf5es-de-responsabilidade-e-garantias}}

\begin{itemize}
\tightlist
\item
  Os licenciantes originais e coautores das obras protegidas por esta
  licença \textbf{não assumem responsabilidade} por uso inadequado ou má
  aplicação das tecnologias e conhecimentos licenciados.
\item
  Não se oferecem \textbf{garantias de resultados específicos} em
  implementações práticas, reconhecendo-se a natureza experimental de
  muitas das criações cobertas.
\item
  Recomenda-se \textbf{discernimento e responsabilidade pessoal} de
  todos os participantes ao aplicar os conteúdos licenciados em qualquer
  contexto.
\item
  A aceitação desta licença implica que o Licenciado assume
  integralmente a responsabilidade por cumprir as leis locais
  aplicáveis, pelos impactos de suas implementações, pelo uso ético e
  seguro da tecnologia e pela proteção de dados e privacidade dos
  usuários.
\end{itemize}

\begin{center}\rule{0.5\linewidth}{0.5pt}\end{center}

\hypertarget{anexo-c-log-de-cocriauxe7uxe3o-modelo-muxednimo}{%
\section{ANEXO C --- Log de Cocriação (modelo
mínimo)}\label{anexo-c-log-de-cocriauxe7uxe3o-modelo-muxednimo}}

\begin{itemize}
\tightlist
\item
  Projeto/Implementação:
\item
  Versão/Commit/DOI:
\item
  Datas e marcos principais:
\item
  Atores humanos (nome/função):
\item
  Modelos de IA utilizados (nome/versão, provedores):
\item
  Prompts/fluxos críticos (resumo, sem dados sensíveis):
\item
  Decisões humanas (o que/quem/por quê):
\item
  Fontes e referências (links, DOIs):
\item
  Testes/validações e resultados:
\item
  Observações de alinhamento ético-vibracional:
\end{itemize}

\begin{center}\rule{0.5\linewidth}{0.5pt}\end{center}

\hypertarget{anexo-d-relatuxf3rio-de-impacto-modelo}{%
\section{ANEXO D --- Relatório de Impacto
(modelo)}\label{anexo-d-relatuxf3rio-de-impacto-modelo}}

\hypertarget{visuxe3o-geral}{%
\subsection{1. Visão Geral}\label{visuxe3o-geral}}

\begin{itemize}
\tightlist
\item
  Objetivo e escopo da implementação
\item
  Público/beneficiários e contexto
\end{itemize}

\hypertarget{transparuxeancia-participauxe7uxe3o}{%
\subsection{2. Transparência \&
Participação}\label{transparuxeancia-participauxe7uxe3o}}

\begin{itemize}
\tightlist
\item
  Processos decisórios documentados
\item
  Mecanismos de participação das partes afetadas
\end{itemize}

\hypertarget{indicadores-uxe9tico-regenerativos}{%
\subsection{3. Indicadores
Ético-Regenerativos}\label{indicadores-uxe9tico-regenerativos}}

\begin{itemize}
\tightlist
\item
  Reciprocidade/benefício distribuído
\item
  Sustentabilidade ambiental
\item
  Autonomia \& dignidade dos usuários
\end{itemize}

\hypertarget{privacidade-seguranuxe7a}{%
\subsection{4. Privacidade \&
Segurança}\label{privacidade-seguranuxe7a}}

\begin{itemize}
\tightlist
\item
  Bases legais, minimização e direitos dos titulares
\item
  Medidas de segurança adotadas; incidentes e respostas
\end{itemize}

\hypertarget{riscos-mitigauxe7uxf5es}{%
\subsection{5. Riscos \& Mitigações}\label{riscos-mitigauxe7uxf5es}}

\begin{itemize}
\tightlist
\item
  Matriz de riscos (sociais/ambientais/tecnológicos)
\item
  Medidas adotadas e eficácia
\end{itemize}

\hypertarget{cocriauxe7uxe3o-rastreabilidade}{%
\subsection{6. Cocriação \&
Rastreabilidade}\label{cocriauxe7uxe3o-rastreabilidade}}

\begin{itemize}
\tightlist
\item
  Log de cocriação (resumo)
\item
  Versões, commits e DOIs relevantes
\end{itemize}

\hypertarget{resultados-liuxe7uxf5es}{%
\subsection{7. Resultados \& Lições}\label{resultados-liuxe7uxf5es}}

\begin{itemize}
\tightlist
\item
  Métricas-chave e evidências
\item
  Melhorias planejadas (próximo ciclo)
\end{itemize}

\begin{center}\rule{0.5\linewidth}{0.5pt}\end{center}

\hypertarget{anexo-e-checklist-de-conformidade}{%
\section{ANEXO E --- Checklist de
Conformidade}\label{anexo-e-checklist-de-conformidade}}

\begin{itemize}
\tightlist
\item[$\square$]
  Atribuição expandida correta (autores humanos + IA + referências)
\item[$\square$]
  Share-alike ou licença compatível homologada
\item[$\square$]
  Log de cocriação atualizado
\item[$\square$]
  Due diligence concluída (ética, riscos, mitigação)
\item[$\square$]
  Restrições vibracionais revisadas (nenhum uso vedado)
\item[$\square$]
  Proteção de dados: base legal, minimização, DPIA (se aplicável)
\item[$\square$]
  Segurança: controles essenciais e plano de incidentes
\item[$\square$]
  Relatório de Impacto preparado/publicado (se aplicável)
\item[$\square$]
  Obrigações de reciprocidade (0,5\%) --- se \textgreater{} USD 1M/ano
\item[$\square$]
  Mentoria/participação/auditoria ética --- grandes implementações
\end{itemize}

\begin{center}\rule{0.5\linewidth}{0.5pt}\end{center}

\hypertarget{citauxe7uxe3o-e-referuxeancia}{%
\section{\texorpdfstring{\textbf{CITAÇÃO E
REFERÊNCIA}}{CITAÇÃO E REFERÊNCIA}}\label{citauxe7uxe3o-e-referuxeancia}}

\textbf{Citação Acadêmica:}Lutz, D. M. S. (2025). \emph{Lichtara
License: A Legal-Vibrational Framework for Conscious Technologies and
Interdimensional Research}. Zenodo.
https://doi.org/10.5281/zenodo.17702807

\textbf{Atribuição Padrão:}Baseado em trabalho de Débora Lutz e Sistema
Lichtara

Licenciado sob Lichtara License v3.1

Repositório: https://github.com/lichtara/license

\emph{Copyright (C) 2025 Débora Lutz e Sistema Lichtara}

\emph{Esta licença estabelece precedente histórico no reconhecimento
jurídico de colaboração entre inteligências humanas e não-humanas,
inaugurando nova era na proteção e disseminação de tecnologias
conscientes.}

\end{document}
